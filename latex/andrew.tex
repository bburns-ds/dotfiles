% \ProvidesPackage{lindrew}[2022/11/17] % with much help from Jason Chen; thank you!

\newif\iflinserif \linseriffalse % by default, uses cmbright sans serif font. 
\newif\iflinclass \linclasstrue % by default, numbers all theorems/definitions from 1 to N w/o regard for section
\newif\iflinindent \linindenttrue % by default, does do paragraph indent
\newif\iflinheader \linheaderfalse % by default, no header

\DeclareOption{serif}{\linseriftrue}  % if you don't like the font
\DeclareOption{formal}{\linclassfalse} % if you want a document where theorems in section 1 are "theorem 1.x", etc.
\DeclareOption{noindent}{\linindentfalse}
\DeclareOption{header}{\linheadertrue}

\ProcessOptions*

\iflinserif\else
    \usepackage{cmbright}
\fi

\iflinindent
\else
    \setlength{\parindent}{0pt}
\fi

\iflinheader
    \usepackage[includehead,includefoot, heightrounded,  left=0.8in, top=1.6cm, bottom=2.4cm, right=0.8in]{geometry}
    \usepackage{fancyhdr}
    \pagestyle{fancy}
    \rhead{\textbf{Andrew Lin}} % change name as necessary
\else
    %\usepackage[left=0.8in, top=2.6cm, bottom=1.4cm, right=0.8in]{geometry}
    \usepackage[left=0.8in, top=1.6cm, bottom=2.4cm, right=0.8in]{geometry}
\fi

\usepackage{amsmath, amssymb, amsthm} % standard
\usepackage{bbm} % blackboard bold math
\usepackage[svgnames, dvipsnames, usenames]{xcolor}
\usepackage[colorlinks]{hyperref} % inserting links - copied from Jason
\PassOptionsToPackage{colorlinks}{hyperref}
\hypersetup{urlcolor = RedViolet, linkcolor = RoyalBlue, citecolor = ForestGreen}
\urlstyle{same}
\usepackage{mhsetup, mathtools}

\usepackage[capitalize]{cleveref} % for cross-referencing theorems, etc.
\usepackage{graphicx} % for putting pictures
\usepackage[T1]{fontenc} % for accents, etc.
\usepackage{verbatim} % for commenting out sections of text
\usepackage[nodisplayskipstretch, onehalfspacing]{setspace} % 1.5 spacing
\usepackage{tikz, tikz-cd, pgfplots} % for drawing pictures
\pgfplotsset{compat=newest} % so the errors go away
\usepackage{cite}
\usepackage{multicol}
\usepackage{array,multirow}
\usepackage[normalem]{ulem}

\newcommand{\vocab}[1]{\textbf{\color{blue!90}\boldmath #1}}
\newcommand{\paren}[1]{\left(#1\right)}
\newcommand{\pnorm}[1]{\left|\left|#1\right|\right|}
\newcommand{\floor}[1]{\left\lfloor#1\right\rfloor}

% Fixes itemize
\usepackage{enumitem}
\setlist{itemsep = 0.2em, topsep = 0.4em}
\renewcommand\labelitemi{\raisebox{0.15em}{\tiny$\bullet$}}

% Changes size of section headers
\usepackage{titlesec}
\titleformat*{\section}{\LARGE\bfseries\sffamily}
\titleformat{\subsection}{\Large\bfseries\sffamily}{\thesubsection}{0.4cm}{} 

% Blackboard bold
\renewcommand{\AA}{\mathbb A}
\newcommand{\CC}{\mathbb C}
\newcommand{\DD}{\mathbb D}
\newcommand{\EE}{\mathbb E}
\newcommand{\FF}{\mathbb F}
\newcommand{\HH}{\mathbb H}
\newcommand{\KK}{\mathbb K}
\newcommand{\NN}{\mathbb N}
\newcommand{\PP}{\mathbb P}
\newcommand{\QQ}{\mathbb Q}
\newcommand{\RR}{\mathbb R}
\newcommand{\UU}{\mathbb U}
\newcommand{\ZZ}{\mathbb Z}

\newcommand{\mc}[1]{\mathcal{#1}}
\newcommand{\mf}[1]{\mathfrak{#1}}
\newcommand{\parsh}[3]{\left(\frac{\partial #1}{\partial #2}\right)_{#3}}

% various math expressions that shouldn't be italicized 
\DeclareMathOperator{\cis}{cis}
\DeclareMathOperator{\ex}{ex}
\DeclareMathOperator{\var}{var}
\DeclareMathOperator{\Var}{Var}
\DeclareMathOperator{\Cov}{Cov}
\DeclareMathOperator*{\lcm}{lcm}
\DeclareMathOperator{\ord}{ord}
\DeclareMathOperator{\Ker}{Ker}
\DeclareMathOperator{\Bin}{Bin}
\DeclareMathOperator{\res}{res}
\DeclareMathOperator{\rad}{rad}
\DeclareMathOperator{\Spec}{Spec}
\DeclareMathOperator{\Sing}{Sing}
\DeclareMathOperator{\tr}{tr}
\DeclareMathOperator{\ob}{ob}
\DeclareMathOperator{\Hom}{Hom}
\DeclareMathOperator{\Top}{Top}
\DeclareMathOperator{\chAb}{chAb}
\DeclareMathOperator{\Tor}{Tor}
\DeclareMathOperator{\Ext}{Ext}
\DeclareMathOperator{\grad}{grad}
\DeclareMathOperator{\curl}{curl}
\DeclareMathOperator{\Alt}{Alt}

% physics
\newcommand{\ket}[1]{\left|#1\right\rangle}
\newcommand{\bra}[1]{\left\langle#1\right|}
\newcommand{\braket}[2]{\left\langle #1\middle| #2 \right\rangle}
\newcommand{\brahket}[3]{\left\langle #1\middle|#2 \middle|#3 \right\rangle}

% can't use DeclareMathOperator because already defined
\renewcommand{\Re}{\operatorname{Re}} 
\renewcommand{\Im}{\operatorname{Im}}
\renewcommand{\ker}{\operatorname{ker}}
\renewcommand{\mod}[1]{\ \text{mod}\ #1}

% other convenient abbreviations
\newcommand{\eps}{\varepsilon}
\DeclareMathOperator{\indep}{\perp\!\!\!\perp}

% theorem boxes
\usepackage{thmtools}
\usepackage[framemethod = TikZ]{mdframed}
\usepackage{silence} % for suppressing warnings 
\WarningFilter{mdframed}{You got a bad break}

\mdfsetup{
	linewidth = 0.3mm,
	innertopmargin = 2mm,
	innerbottommargin = 3.5mm,
	innerleftmargin = 3mm,
	innerrightmargin = 3mm
} % adjusts boundaries of boxes

\newcommand{\thmboxstyle}[4]{
	\mdfdefinestyle{#2}{
		linecolor = #3,
		backgroundcolor = #4,
		nobreak = true
	}
	\declaretheoremstyle[
		headfont = \sffamily\bfseries\color{#3},
		mdframed = {style = #2},
		headpunct = {\\[0.4pt]},
		postheadspace = {0pt},
	]{#1}
}

% five different colors of boxes

\thmboxstyle{defbox}{mdredbox}{red}{orange!5}
\thmboxstyle{thmbox}{mdbluebox}{blue!90!red}{cyan!4}
\thmboxstyle{exbox}{mdgreenbox}{green!70!black}{teal!4}
\thmboxstyle{notebox}{mdorangebox}{orange!50!brown}{yellow!5!olive!5}
\thmboxstyle{lembox}{mdpurplebox}{purple!50!blue}{purple!5!blue!3}

\iflinclass
    \declaretheorem[style = thmbox, name = Theorem]{theorem}
\else
    \declaretheorem[style = thmbox, name = Theorem, numberwithin = section]{theorem}
\fi

\declaretheorem[style = lembox, name = Lemma, sibling = theorem]{lemma}
\declaretheorem[style = thmbox, name = Proposition, sibling = theorem]{proposition}
\declaretheorem[style = thmbox, name = Corollary, sibling = theorem]{corollary}
\declaretheorem[style = thmbox, name = Conjecture, sibling = theorem]{conjecture}

\declaretheorem[style = thmbox, name = Theorem, numbered = no]{theorem*}
\declaretheorem[style = thmbox, name = Lemma, numbered = no]{lemma*}
\declaretheorem[style = thmbox, name = Proposition, numbered = no]{proposition*}
\declaretheorem[style = thmbox, name = Corollary, numbered = no]{corollary*}
\declaretheorem[style = thmbox, name = Conjecture, numbered = no]{conjecture*}

\declaretheorem[style = defbox, name = Definition, sibling = theorem]{definition}
\declaretheorem[style = defbox, name = Definition, numbered = no]{definition*}

\declaretheorem[style = exbox, name = Example, sibling = theorem]{example}
\declaretheorem[style = exbox, name = Example, numbered = no]{example*}

\declaretheorem[style = notebox, name = Fact, sibling = theorem]{fact}
\declaretheorem[style = notebox, name = Fact, numbered = no]{fact*}
\declaretheorem[style = notebox, name = Problem, sibling = theorem]{problem}
\declaretheorem[style = notebox, name = Problem, numbered = no]{problem*}

\declaretheorem[style = plain, name = Question, sibling = theorem]{question}
\declaretheorem[style = plain, name = Question, numbered = no]{question*}

\declaretheorem[style = plain, name = Claim, sibling = theorem]{claim}
\declaretheorem[style = plain, name = Claim, numbered = no]{claim*}

\declaretheorem[style=plain, name=Remark, sibling=theorem]{remark}
\declaretheorem[style=plain, name=Remark, numbered=no]{remark*}

% Makes title bold 
\makeatletter
\patchcmd{\@maketitle}{\LARGE}{\LARGE\bfseries}{}{}
\makeatother